\documentclass[10pt,a4paper,sans]{moderncv}
\moderncvstyle{classic}
\moderncvcolor{blue}
%\nopagenumbers{}
\usepackage[utf8]{inputenc}
\usepackage[scale=0.92]{geometry}
\setlength{\hintscolumnwidth}{3cm}
\name{Maxime Ulysse}{Garcia}
\title{PhD in Bioinformatics}
\extrainfo{Date of birth: 05/14/1982\\Nationality: French}
% \address{Markvardsgatan 1}{113 53 Stockholm}{Sweden}
\phone[mobile]{+46-723734336}
\email{max@ithake.eu}
\homepage{ithake.eu/}
% \social[linkedin]{maxugarcia}
% \social[twitter]{gau}
% \social[github]{MaxUlysse}
\quote{Specialized in large scale data integration and visualization}

\begin{document}
	\makecvtitle
	\section{Experiences}
		\cventry{2016--present}{Bioinformatician}{}{BarnTumörBanken - Sitting at both Karolinska Institutet / Instutitionen För Onkologi-Patologi and at Science for Life Laboratory / NGI Stockholm (Genomics Applications)}{Stockholm}{}
		\cventry{2013--2014}{Research Associate in \textit{Bioinformatics}}{}{National University of Singapore - Cancer Science Institute of Singapore}{Singapore}{}
		\cvitem{experiences}{\small{Intern management; Collaboration with PhDs, Engineers; Presentations in international conferences (posters and talks)}}
		\cventry{2009--2013}{PhD in \textit{Bioinformatics and Genomics}}{}{Integrative Bioinformatics Platform - Centre de Recherche en Cancérologie de Marseille, Aix-Marseille Université UM105, Intitut Paoli-Calmettes, Inserm U1028, CNRS UMR7258}{Marseille}{Biomarkers discovery in cancer by Interactome-Transcriptome Integration}
		\cvitem{experiences}{\small{Intern management; Collaboration with MDs, PhDs, Research Engineers and socio-economic partner (IPSOGEN); Presentations in international conferences (posters and talks)}}
		\cvitem{skills}{\small{Parallel scripting; Project management; Oral and visual communication}}

	\section{Education \& Qualifications}
		\cventry{2007--2009}{Master's Degree in \textit{Bioinformatics and Genomics}}{}{Faculté des Sciences de Luminy}{Marseille}{}
		\cvitem{keywords}{\small{Object-oriented programming, Dynamic modelization of biological networks, Mathematics applied to Biology}}
		\cventry{2007}{Erasmus Semester in \textit{School of Biological Sciences}}{}{University of East Anglia}{Norwich}{}
		\cvitem{keywords}{\small{Research project in Bioinformatics}}
		\cventry{2003--2007}{Bachelor's Degree in \textit{Biology and Biochemistry}}{}{Faculté des Sciences de Luminy}{Marseille}{}

	\section{Publications}
		\cvitem{chapter}{M. Garcia et al., CNV-Interactome-Transcriptome Integration to detect driver genes in cancerology, \textit{Microarray Image and Data Analysis: Theory and Practice}, CRC Press, 2014}
		\cvitem{article}{T. Bonacci et al., Chemotherapy induced changes of post-translational modifications by ubiquitin and ubiquitin-like proteins in pancreatic cancer cells, \textit{J Proteome Res.} 2014 May 2;13(5):2478-94}
		\cvitem{chapter}{M. Garcia et al., Detection of driver protein complexes in breast cancer metastasis by large scale transcriptome-interactome integration, \textit{Methods Mol Biol.} 2014;1101:67-85}
		\cvitem{article}{M. Garcia et al., Interactome-Transcriptome integration for predicting distant metastasis in breast cancer, \textit{Bioinformatics}, 2012}
		\cvitem{chapter}{M. Garcia et al., Linking interactome to disease: a network-based analysis of metastatic relapse in breast cancer \textit{Handbook of Research on Computational and Systems Biology: Interdisciplinary Applications}, IGI Global, 406-427, 2011}

	\section{Computers skills}
		\cvdoubleitem{Scripting}{Perl, Python, Bash}{Statistics}{MATLAB, R}
		\cvdoubleitem{Web}{TypeScript, JavaScript, PHP/MySQL}{CMS}{Wordpress, Joomla}
		\cvdoubleitem{Management}{Trello, Redmine, SVN, Git}{Writing}{\LaTeX, Office Suite}
		\cvdoubleitem{Graphism}{Inkscape}{+}{Use of cluster, Linux}

	\section{Languages}
		\cvitem{French}{Mother tongue}
		\cvitem{English}{Fluent}

	\clearpage

	\section{Presentations}
		\cvitem{2016}{The XVth KICancer Retreat 2016 - \textit{poster}, Djurö, Sweden}
		\cvitem{2013}{5th Frontiers in Cancer Science (FCS) - \textit{poster}, Singapore, Singapore}
		\cvitem{2012}{11th European Conference on Computational Biology (ECCB) - \textit{poster}, Basel, Switzerland}
		\cvitem{2012}{13ème Journées Ouvertes en Biologie Informatique et Mathématiques (JOBIM) - \textit{poster}, Rennes, France}
		\cvitem{2012}{6ème Colloque du Cancéropôle PACA - \textit{poster}, Marseille, France}
		\cvitem{2011}{12ème Journées Ouvertes en Biologie Informatique et Mathématiques (JOBIM) - \textit{poster}, Paris, France}
		\cvitem{2011}{19th meeting of doctoral school - \textit{poster}, Marseille, France}
		\cvitem{2011}{Mathematical and Statistical Aspects of Molecular Biology (MASAMB) XXI, Vienna, Austria}
		\cvitem{2010}{Biologistes Chimistes et Physiciens VII - \textit{poster}, Marseille, France}
		\cvitem{2010}{Cancer Bioinformatics Workshop, Cambridge, UK}
		\cvitem{2010}{Interactomics: at the crossroads of biology and bioinformatics - \textit{poster}, St Raphaël, France}

	\section{Fellowhips}
		\cvitem{2012}{ECCB’12 Conference Fellowship from SIB Swiss Institute of Bioinformatics}
		\cvitem{2011}{Travel grant from MASAMB organisation}
		\cvitem{2009}{PhD fellowhip from INSERM/Région PACA}

	\section{Other Responsibilities}
		\cvitem{Scientific Society}{Member of International Society of Computational Biology and Société Française de BioInformatique}
		\cvitem{Associations}{Vice-president of JeBiF - RSG-France (Association of French Young Bioinformaticians) from 2012 to 2014}
		\cvitem{Doctoriales}{Member of organization committee and logistic support for Doctoriales en Provence in 2012}
		\cvitem{Fête de la Science}{Conception and animation of a stand at Marché de la Chimie for Cancéropôle-PACA in 2011}
		\cvitem{Blogs}{Contributor to \url{http://bioinfo-fr.net/} and \url{http://www.leblogducinema.com/}}
		\cvitem{Conventions}{Logistic support in organization of Plan de Cuques Comics Festival 2007 to 2011}
		\cvitem{Skill}{Sauveteur-Secouriste du Travail (French First-Aid Workplace Diploma)}

	\section{Interests}
		\cvitem{}{Taï-Chi-Chuan, Gardening, Roller}

\end{document}